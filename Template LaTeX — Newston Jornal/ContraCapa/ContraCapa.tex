\newpage
\thispagestyle{empty}
% Linhas verticais Contra Capa
\begin{tikzpicture}[remember picture, overlay]
    \draw[line width=3.1cm, base] ($(current page.north east) - (0,0)$) -- ($(current page.south east) - (0,0)$);
    \draw[line width=1.05cm, base] ($(current page.north east) - (2.25,0)$) -- ($(current page.south east) - (2.25,0)$);
    \draw[line width=22.6cm, base] ($(current page.north east) - (14.25,0)$) -- ($(current page.south east) - (14.25,0)$);
\end{tikzpicture}

% Textos
\begin{tikzpicture}[remember picture, overlay]
  % Adiciona texto
  \node at (current page.north west) [anchor=north west, xshift=2cm, yshift=-3cm] {\begin{minipage}[t]{8cm} % Define a largura da minipage igual à largura da figura
  \setlength{\baselineskip}{1.5\baselineskip}
    {\sffamily\myfontsizeTemaCapaVerticalECC O Newston Jornal surgiu como uma iniciativa estudantil envolvendo alunos dos cursos de Física, Letras e Arquitetura e Urbanismo da Universidade Estadual de Maringá (UEM) em 2018, e foi oficializado como projeto de extensão vinculado oficialmente à universidade em março de 2021.} \end{minipage}
  };
   \node at (current page.north west) [anchor=north west, xshift=2cm, yshift=-10cm] {\begin{minipage}[t]{8cm} % Define a largura da minipage igual à largura da figura
  \setlength{\baselineskip}{1.5\baselineskip}
    {\sffamily\myfontsizeTemaCapaVerticalECC Desde o início, a finalidade do projeto consiste em produzir material para divulgação científica e cultural que atenda à comunidade externa e interna da universidade, visando sempre a integração de diversas áreas do conhecimento.} \end{minipage}
  };
\end{tikzpicture}

% Logo do Newston e da UEM
\begin{tikzpicture}[remember picture, overlay]
  % Adiciona a logo maçã branco
  \node at (current page.north west) [anchor=north west, xshift=1.8cm, yshift=-25.2cm] {
    \includegraphics[width=6cm]{ContraCapa/Newston_Jornal_Logo_Horizontal_ContraCapa.png}
  };
  % Adiciona a logo da UEM Branco
  \node at (current page.north west) [anchor=north west, xshift=12.5cm, yshift=-25.7cm] {
    \includegraphics[width=0.22\textwidth]{Capa/Figuras_Capa/UEM_Logo_Modelo_Branco.png}
  };
\end{tikzpicture}