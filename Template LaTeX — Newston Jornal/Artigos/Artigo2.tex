% SEGUNDO ARTIGO
\newcommand{\titletwo}{Título Principal - Artigo 2} 
\mytitle{\titletwo}    
\addchaptersummary{\titletwo}{Sumario/Figs_Sumario/FigArtigo2.jpg}{Tintim é o protagonista de As Aventuras de Tintim, criada pelo quadrinista belga Hergé que estreou em 10 de janeiro de 1929.}{Tintim} 

\newcommand{\refarticletwo}{\titletwo} 

\begin{multicols}{2}

Para adicionar uma figura no \textit{layout} da página inteira, ou seja em uma coluna, basta fechar o ambiente \texttt{multicols} temporariamente. Use o comando \texttt{\textbackslash par \textbackslash noindent} para garantir que a imagem fique corretamente alinhada e evitar espaçamentos indesejados quando saímos do ambiente \texttt{multicols} e em seguida insira a figura com o ambiente \texttt{figure}. Exemplo a seguir.

\end{multicols}

\par\noindent
\begin{figure}[H]
    \centering
    \includegraphics[width=\linewidth]{Figuras/Artigo2/Tintim.jpg}
    \caption{Tintim e seu inseparável amigo de 4 patas Bilu. Fonte: \href{https://br.pinterest.com/}{Pinterest}.}
    \label{fig:tintim}
\end{figure}


\begin{multicols}{2}
Para voltar ao \textit{layout} de duas colunas reinicie o ambiente \texttt{\textbackslash begin\{multicols\}\{2\}}. 

\lipsum[1-3]

\authorinfo{Tintim}{link lattes ou orcid}

\end{multicols}