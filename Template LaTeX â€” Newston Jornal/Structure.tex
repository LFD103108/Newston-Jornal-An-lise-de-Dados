% PACOTES, NOVOS COMANDOS E OUTRAS ESPECIFICAÇÕES
\usepackage{graphicx}
%\usepackage{xcolor}
\usepackage[table]{xcolor}
\usepackage[small]{caption}
%\usepackage[caption=false]{subfig}
\usepackage{subfigure}
\usepackage{float} % Para usar [H]
%\usepackage{bbm}
\usepackage{amsmath, amssymb}
\usepackage[T1]{fontenc}
%\usepackage[utf8]{inputenc}
%\usepackage[portuguese]{babel}
\usepackage[brazilian]{babel} %Indica a língua a ser escrita
\usepackage[normalem]{ulem}
%\usepackage[left=00cm, right=00cm, top=0.5cm, bottom=1.5cm]{geometry}
\usepackage{parskip}
\usepackage{color}
\usepackage{colortbl}
\usepackage{array}
\usepackage{setspace}
\usepackage{minted}
\usepackage{hyperref} 
\hypersetup{colorlinks=true, citecolor=blue, linkcolor=blue, urlcolor=blue}
	\usemintedstyle{perldoc}
\usepackage{multirow}
\usepackage{fancyhdr}
\usepackage{multicol}
\usepackage{balance} % Para equilibrar as colunas
\usepackage{indentfirst}
\usepackage{epstopdf}
\usepackage{animate}
\usepackage{xmpmulti}
\usepackage{multimedia}
\usepackage{wrapfig}
\usepackage{setspace}
\usepackage{pdfpages}
\usepackage{braket}


%------------------------------------------------------------------
% Cores - Sugestão: use uma cor parecida com a cor base (indico o site https://www.colorhexa.com/ para verificar os códigos de acordo com a cor da sua preferência)
\definecolor{base}{HTML}{4F3B89} % Cor hexadecimal da edição (altere de acordo com a sua preferência)  EX: 893B4E 
\definecolor{baseshadow}{HTML}{674DB2} % Cor hexadecimal do sombreamento dos textos na edição 
\definecolor{baseechoshadow}{HTML}{C4BAE1} % Cor hexadecimal do sombreamento do sombreamento dos textos na edição 
\setlength{\columnsep}{1cm}

%------------------------------------------------------------
% Comando para títulos de artigos
\usepackage{tikz}
\usetikzlibrary{calc} %Bibliotecas TikZ
\newcommand{\mytitle}[1]{%
    \newpage
    \phantomsection
    \begin{LARGE}
        \begin{center}
            \textcolor{base}{\textit{#1}}
        \end{center}
    \end{LARGE}
     \markboth{#1}{#1} % Define o marcador do cabeçalho
}
% Comando para subtítulos de títulos de artigos
\newcommand{\mytitlesubtitle}[1]{%
    \vspace{-.2cm}
    \begin{Large}
        \begin{center}
            \textcolor{base}{\textit{#1}}
        \end{center}
    \end{Large}
}
% Comando para subtítulos de artigos
\newcommand{\mysubtitle}[1]{%
       \phantomsection
       \vspace{5pt} % Espaço acima do subtítulo
       \begin{flushleft}
            \textcolor{base}{\textbf{#1}}
       \end{flushleft}
     \vspace{5pt} % Espaço abaixo do subtítulo
}

% Configurando o cabeçalho
\pagestyle{fancy}
\fancyhf{}

% Redefinindo a linha do cabeçalho com a cor e largura desejadas
\renewcommand{\headrule}{%
    \color{base} % Define a cor da linha
    \hrule width\headwidth height4pt % Define a largura da linha
    \vskip-\headrulewidth % Eleva a linha
}

% Garantir que a linha cubra toda a largura da página
\fancyheadoffset[L]{\dimexpr\hoffset+2cm\relax} % Alinhar à esquerda com a margem esquerda
\fancyheadoffset[R]{\dimexpr\hoffset+2cm\relax} % Alinhar à direita com a margem direita
\fancyhead[L]{\textcolor{base}
{\MakeUppercase{\large{\bf\hspace{0.3cm}\leftmark}}\vspace{0.35cm}}}
\setlength{\headheight}{33pt} % Ajuste o valor conforme necessário para consertar o erro headheight quando surgir
\setlength{\headsep}{35pt} % Ajusta a distância entre o cabeçalho e o texto
\fancyfoot[C]{\thepage} % Numeração das páginas
%\usepackage{flushend} % Pode ajudar a corrigir problemas de alinhamento
% Definindo um estilo sem cabeçalho, mas com numeração de página
\fancypagestyle{noheader}{
  \fancyhf{} % Limpa cabeçalhos e rodapés
  \renewcommand{\headrule}{%
    \color{base} % Define a cor da linha
    \hrule width\headwidth height0pt % Exlui a linha do cabeçalho
    \vskip-\headrulewidth % Eleva a linha
}
  \fancyfoot[C]{\thepage} % Mantém o rodapé com numeração de página
}


\usepackage{lipsum}
%\usepackage{indentfirst}
\usepackage{anysize} % Para personalizar a largura das margens
%\usepackage[margin=0.7874in]{geometry} % Define as margens
\marginsize{2cm}{2cm}{0.5cm}{2cm} % Ordem: Esquerda, direita, superior, inferior. Padrão ABNT - Esquerda: 3 cm. Superior: 0.5 cm. Direita: 2 cm. Inferior: 2 cm.
%\usepackage{geometry}
\usepackage{marginnote}
\setlength{\parindent}{20pt} %Indentação do parágrafo
\usepackage{subcaption}
\usepackage{subfigure}

%FONTES
\usepackage{anyfontsize} % Permite o uso de tamanhos de fonte arbitrários, independentemente dos tamanhos padrão
\usepackage{microtype} % Ajuste de espaçamento entre letras
%\usepackage{fontspec} % Pacote para usar fontes TrueType e OpenType (só funciona compilando em LuaLaTeX)
\usepackage{enumitem} % Pacote para personalizar listas
\usepackage{shadowtext} % Pacote para adicionar sombras ao texto
\usepackage{setspace} % Pacote para aumentar ou diminuir o espaço entre linhas
\usepackage{contour} % Para contornos em textos


% Comando personalizado para texto com sombra
\newcommand{\shadowedText}[2][black]{%
    \begin{tikzpicture}[baseline=(T.base)]
        \node[inner sep=0pt, outer sep=0pt, text=#1] (S) at (0.06, -0.06) {#2}; % Ajuste (x, y) conforme necessário para mudar a posição da sombra em relação ao texto.
        \node[inner sep=0pt, outer sep=0pt, text=white] (T) at (0, 0) {#2};
    \end{tikzpicture}%
}

% Comando para texto com sombra e efeito eco
\newcommand{\echoShadowText}[3]{%
    \begin{tikzpicture}[baseline=(T.base)]
        % Sombra principal com deslocamento maior
        \node[inner sep=0pt, outer sep=0pt, text=#1, opacity=0.6] (S1) at (0.08, -0.08) {#3};
        % Sombra secundária com deslocamento menor
        \node[inner sep=0pt, outer sep=0pt, text=#2, opacity=0.6] (S2) at (0.04, -0.04) {#3};
        % Texto principal
        \node[inner sep=0pt, outer sep=0pt, text=white] (T) at (0, 0) {#3};
    \end{tikzpicture}%
}

%
\usepackage{titlesec}
\usepackage{titletoc}

% Define o estilo dos títulos do sumário
\titlecontents{chapter}
  [0em] % Margem esquerda
  {\sffamily\bfseries} % Fonte e estilo do título
  {\contentslabel{2em}} % Largura da etiqueta
  {} % Texto à esquerda do título
  {\hspace{0.1em}\titlerule*[0.4pc]{.}\hspace{-0.3em}\contentspage} % Linha de pontos até o número da página  
  [\vspace{0.4em}] % Espaço abaixo do título

% Definindo um novo comando \newauthor para o Sumário
\newcommand{\newauthor}[2]{{#1}{#2}}

% Comando personalizado para adicionar um capítulo ao sumário com título, imagem e resumo
\newcommand{\addchaptersummary}[4]{
  \phantomsection
  \addcontentsline{toc}{chapter}{\MakeUppercase{#1}}
  \addtocontents{toc}{
    \protect\begin{minipage}[t][.7cm][t]{4cm} % Alinhamento superior
      \protect\includegraphics[width=6em, height=6em]{#2}
    \protect\end{minipage}%
    \protect\begin{minipage}[t][.3cm][t]{9cm}% Definindo largura fixa para a minipage de texto com alinhamento superior
      \protect\vspace{-6em} % Ajuste fino para alinhar o resumo do texto com o topo da figura (altere de acordo com o topo da figura)
      {\normalsize #3}\\
      \protect\vfill % Espaço flexível que empurra o texto do autor para baixo
      \protect\vfill% Ajuste fino para alinhar o nome do autor do texto com a base da figura (altere de acordo com a base da figura)
      \normalsize{\newauthor{\textbf{Autor(a):}}{\,#4}}
      \protect\vspace{.2cm} % Ajuste fino para espaçamento entre figura e resumo e próximo artigo do sumário
    \protect\end{minipage}\vfill}
}

%[t]{7em}
% Remove o título "Sumário"
\pretocmd{\tableofcontents}{\renewcommand{\contentsname}{}}{}{\undefined}
% Redefinir o título da bibliografia
\usepackage{etoolbox}
\usepackage{adjustbox}
\usepackage{bookmark} % Ajuda na criação de links corretos
% Redefine a cor dos links no sumário
\AtBeginDocument{
    \pretocmd{\tableofcontents}{\hypersetup{linkcolor=black}}{}{}
    \apptocmd{\tableofcontents}{\hypersetup{linkcolor=blue}}{}{}
}

% Definindo o comando \authorinfo
\newcommand{\authorinfo}[2]{\vspace{1cm}\noindent%
    \textbf{\small{\textit{Escrito por: \href{#2}{#1}}}}
}
\usepackage{microtype}

% Redefine the numbering format for enumerate
\renewcommand{\theenumi}{[\arabic{enumi}]}
\renewcommand{\labelenumi}{\theenumi\ }
% Adjust the spacing between the number and the text
\setlist[enumerate,1]{left=0em, labelsep=.3em}

% Comandos de Referências

% Define a command to combine multiple references into one set of brackets
\newcommand{\refdois}[2]{%
  \textcolor{blue}{[}\ref{#1}%
  \ifx&#2&\else
    \textcolor{blue}{,}\ref{#2}%
  \fi
  \textcolor{blue}{]}%
}

\newcommand{\reftres}[3]{%
  \textcolor{blue}{[}%
  \ifx&#1&%
    \else \ref{#1}%
  \fi%
  \ifx&#2&%
    \else \ifx&#1&%
      \ref{#2}%
    \else
      \textcolor{blue}{,}\ref{#2}%
    \fi%
  \fi%
  \ifx&#3&%
    \else \ifx&#1&\ifx&#2&%
      \ref{#3}%
    \else
      \textcolor{blue}{,}\ref{#3}%
    \fi%
    \else
      \textcolor{blue}{,}\ref{#3}%
    \fi%
  \fi%
  \textcolor{blue}{]}%
}



\renewcommand{\sfdefault}{phv} % Define Helvetica (similar à Arial) como fonte sans-serif
\usepackage[scaled]{helvet} % Usa Helvetica com escala ajustável

%\usepackage{tgadventor}






% Definir um comando para usar fontes com tamanhos e cores personalizadas

% Página Capa
\newcommand{\myfontsizeTema}{\fontsize{40pt}{0pt}\selectfont\color{white}}
\newcommand{\myfontsizeTituloTema}{\fontsize{76pt}{0pt}\selectfont\color{white}}
\newcommand{\myfontsizeTitulosArtigosCapa}{\fontsize{25pt}{0pt}\selectfont\color{white}}
\newcommand{\myfontsizeTitulosArtigosCapaDois}{\fontsize{20pt}{0pt}\selectfont\color{white}}

% Página Folha Rosto
\newcommand{\myfontsizeColecaoData}{\fontsize{11pt}{0pt}\selectfont\color{base}}
\newcommand{\myfontsizeColecaoTema}{\fontsize{18pt}{0pt}\selectfont\color{gray}}
\newcommand{\myfontsizeEditionFolhaRosto}{\fontsize{18pt}{0pt}\selectfont}
\newcommand{\myfontsizeDivulgacaoCientifica}{\fontsize{45pt}{0pt}\selectfont\color{white}}

% Página Sumário
\newcommand{\myfontsizeSumario}{\fontsize{72pt}{0pt}\selectfont\color{base}}
\newcommand{\myfontsizeSdeSumario}{\fontsize{92pt}{0pt}\selectfont\color{base}}
\newcommand{\myfontsizeSumarioData}{\fontsize{27pt}{0pt}\selectfont\color{base}}
\newcommand{\myfontsizeSumarioTitulos}{\fontsize{16pt}{0pt}\selectfont}
\newcommand{\myfontsizeEdAntSite}{\fontsize{9pt}{0pt}\selectfont\color{white}}
\newcommand{\myfontsizeUsuario}{\fontsize{10pt}{0pt}\selectfont\color{white}}

% Página Equipe
\newcommand{\myfontsizeequipe}{\fontsize{42.6pt}{48pt}\selectfont}
\newcommand{\myfontsizecolabbebas}{\fontsize{20pt}{48pt}\selectfont}
\newcommand{\myfontsizeEdition}{\fontsize{21.3pt}{48pt}\selectfont}
\newcommand{\myfontsizeColaboradores}{\fontsize{12.6pt}{48pt}\selectfont}

% Página ContraCapa
\newcommand{\myfontsizeTemaCapaVerticalECC}{\fontsize{15pt}{0pt}\selectfont\color{white}}









