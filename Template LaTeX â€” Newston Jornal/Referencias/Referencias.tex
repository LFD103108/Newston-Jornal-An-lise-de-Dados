% Título e referências na mesma página
\mytitle{Referências}% Título da seção de referências[]
\vspace{.5cm}
% Use \ref{ref1:artigo1} para citar uma referência no texto ou \refdois{ref1:artigo1}{ref2:artigo1} e \reftres{ref1:artigo1}{ref2:artigo1}{ref3:artigo1} para citar duas ou três referências.. para mais de três crie um ambiente parecido com o que define esses novos comandos em Structure.tex
\begin{center}
    \textcolor{base}{\MakeUppercase{\refarticleone}} % Não se esqueça de alterar \refarticleone para \refarticletwo e assim por diante para referências de outros artigos da edição
\end{center}

% Ambiente de referências, adicione uma referência para cada item. Não se esqueça de alterar a entrada em \label toda vez que for adicionar uma nova referência
\begin{enumerate}
    \item \label{ref1:artigo1} GUTH, Alan H. \textit{The Inflationary Universe: The Quest For a New Theory of Cosmic Origins}. [S.l.]: Helix Books, 1997. v. 33-57, Chapter 3: The birth of modern cosmology, p. 3357.
    \item \label{ref2:artigo1} RYDEN, Barbara. \textit{Introduction to cosmology}. [S.l.]: Cambridge University Press, 2017. 5.
    \item \label{ref3:artigo1} WEINBERG, Steven. \textit{Cosmology.} Oxford: Oxford University Press, 2008.
    \item \label{ref4:artigo1} DODELSON, Scott. \textit{Modern Cosmology.} Academic Press, 2003.
\end{enumerate}

% Referências Artigo 2
\begin{center}
    \textcolor{base}{\MakeUppercase{\refarticletwo}}
\end{center}

\begin{enumerate}
    \item \label{ref1:artigo2} BARLOW, R. J. Statistics: A Guide to the Use of Statistical Methods in the Physical Sciences. Wiley, 1989.
    \item \label{ref3:artigo2} GRIFFITHS, David J. \textit{Introduction to Electrodynamics.} 4. ed. Cambridge: Cambridge University Press, 2017.
\end{enumerate}

% Referências Artigo 3
\begin{center}
    \textcolor{base}{\MakeUppercase{\refarticlethree}}
\end{center}

\begin{enumerate}
    \item \label{ref1:artigo3} SAKURAI, J. J.; NAPOLITANO, Jim. Modern Quantum Mechanics. 2. ed. Cambridge: Cambridge University Press, 2017.
\end{enumerate}

% Referências para as figuras da capa e sumário

\begin{center}
    \textcolor{base}{\MakeUppercase{Figuras da Capa e Sumário}}
\end{center}

Figuras da capa: \href{https://www.britannica.com/science/cosmic-microwave-background}{CMB} e \href{https://www.canva.com/}{Fundo da capa};

Figura da contra-capa: \href{https://www.eso.org/public/images/eso1205a/}{Nebulosa Helix - EOS};

Figuras do sumário:

\begin{itemize}
    \item[i)] \href{https://pt.wikipedia.org/wiki/Steamboat_Willie}{\titleone};
    \item[ii)] \href{https://br.pinterest.com/}{\titletwo};
    \item[iii)] \href{https://br.pinterest.com/}{\titlethree};
\end{itemize}




